% based on https://github.com/ivanp7/bgu-rfikt-diplom/
% P.S. probably can remove many unneeded usepackages...

% split long urls on hyphenate. Also then it will never ever split words by
% hyphen inside url
\PassOptionsToPackage{hyphens}{url}
% color names
\PassOptionsToPackage{svgnames}{xcolor}

\documentclass[12pt, a4paper]{extreport}

\newcommand{\comment}[1]{}

\usepackage{graphicx}
\usepackage{xcolor}
\usepackage[utf8]{inputenc}
\usepackage[T1, T2A]{fontenc}
\usepackage{fixltx2e}
\usepackage{grffile}
\usepackage{longtable}
\usepackage{wrapfig}
\usepackage{rotating}
\usepackage[normalem]{ulem}
\usepackage{amsmath}
\usepackage{textcomp}
\usepackage{amssymb}
\usepackage{capt-of}
\usepackage{totcount}
\usepackage[figure,table]{totalcount}
\usepackage{etoolbox}
\usepackage{tocvsec2}
\usepackage{booktabs}

\setcounter{secnumdepth}{2}

\regtotcounter{page}

\newtotcounter{citenum}
\def\oldcite{}
\let\oldcite=\bibcite
\def\bibcite{\stepcounter{citenum}\oldcite}

\newtotcounter{attachcnt}

\usepackage{indentfirst}
\usepackage[left=3cm,right=2cm,
  top=2cm,bottom=2cm,bindingoffset=0cm]{geometry}
\usepackage[nodisplayskipstretch]{setspace}
\onehalfspacing
%\parindent=1cm

\usepackage{enumitem}
\setlist{nolistsep}

\usepackage{titlesec}
\titleformat{\chapter}[display]{\filcenter\bfseries\Large}{}{8pt}{}{}
\titleformat{\section}{\filcenter\bfseries\large}{\thesection}{1em}{}{}
\titleformat{\subsection}{\filcenter\bfseries\normalsize}{\thesubsection}{1em}{}{}
\titleformat{\subsubsection}{\filcenter\bfseries\normalsize}{}{1em}{}{}

% Настройка вертикальных и горизонтальных отступов
\titlespacing*{\chapter}{0pt}{-30pt}{*2}
\titlespacing*{\section}{\parindent}{*2}{*1}
\titlespacing*{\subsection}{\parindent}{*2}{*1}
\titlespacing*{\subsubsection}{\parindent}{*0.5}{*0}

\usepackage{polyglossia}
\setdefaultlanguage[spelling=modern]{russian}
\setotherlanguage{english}
\defaultfontfeatures{Mapping=tex-text}
\newfontfamily{\cyrillicfont}{Times New Roman}
\newfontfamily{\cyrillicfonttt}{Courier New} % шрифт URL-ссылок
%\newfontfamily{\sourcecodefont}{Courier New}
\defaultfontfeatures{Ligatures={TeX},Renderer=Basic}    %% свойства шрифтов по умолчанию
\setmainfont[Ligatures={TeX,Historic}]{Times New Roman} %% задаёт основной шрифт документа
%% \setsansfont{CMU Sans Serif}                         %% задаёт шрифт без засечек
\setmonofont{Courier New}                               %% задаёт моноширинный шрифт

\newcommand{\fixedspaceword}[2][1]{%
  \begingroup
  \spaceskip=#1\fontdimen2\font
  \xspaceskip=0pt\relax % just to be sure
  #2%
  \endgroup
}

% объявляем новую команду для переноса строки внутри ячейки таблицы
\newcommand{\spcell}[2][c]{%
  \begin{tabular}[#1]{@{}c@{}}#2\end{tabular}}

% \bibliographystyle{gost705}

\makeatletter
\def\@biblabel#1{#1. }
\makeatother

\AtBeginDocument{%
  \def\contentsname{ОГЛАВЛЕНИЕ}
  \def\bibname{СПИСОК ИСПОЛЬЗОВАННОЙ ЛИТЕРАТУРЫ}
  \renewcommand{\figurename}{Рисунок}
  \renewcommand{\tablename}{Таблица}
  % style of numbers in captions. Can include chapter/section number, etc.
  \renewcommand{\thefigure}{\arabic{figure}}
  \renewcommand{\thetable}{\arabic{table}}
  % \renewcommand{\theequation}{\arabic{chapter}.\arabic{equation}}
}
